% !TEX root = thesis_draft.tex

\section{General Introduction}
%``If I have seen further it is by standing on the shoulders of
%Giants.'' -- Isaac Newton\\
%``The only thing that will redeem mandkind is co-operation.'' --
%Bertrand Russell\\~\\

What happens in the brain when we work together? It is a worthwhile question, as
social coordination plays a central role in our everyday lives. A better
understanding of it could help us compete and cooperate more efficiently, e.g.
during negotiations, pair programming or construction projects. Social
coordination is also key in more mundane joint actions. For example, carrying a
heavy object together \parencite{sebanz_joint_2006}. If we can model what
happens in the brain during social interaction mathematically, as proposed by
\textcite{koike_hyperscanning_2015}, it could even help us build Human-Computer
Interaction systems that better anticipate the user's needs \parencite{tan_enhancing_2010}.

The classical way of researching social coordination is to study a single
participant in a lab environment
\parencite{hasson_brain--brain_2012,babiloni_social_2014}. A good example of
this is the false belief task, in which the participant is told a story in which
the environment changes without the knowledge of an observer
\parencite[p.~458]{postle_essentials_2020}. Afterwards, the participant is asked
about the observer's beliefs on the current state of the environment. It is
commonly used to study theory of mind\footnote{See Appendix~\ref{app:tom} for
more information on theory of mind.}, i.e. the ability to anticipate other
people's behaviour \parencite[p.~457]{postle_essentials_2020}. Within this
research approach, brain imaging is often used to find the cerebral regions that
are involved in performing the task \parencite{babiloni_social_2014}.

While this classical approach has been very succesful, it has its limits too.
Humans are known to behave differently when not interacting with an actual
person \parencite{babiloni_social_2014,rilling_neural_2004,rilling_neuroscience_2011}.
The approach also cannot be used to study reactions that arise dynamically, i.e.
spontaneously, as a result of information exchanged during the social
interaction of interest \parencite{babiloni_social_2014, czeszumski_hyperscanning_2020}.

\subsection{Hyperscanning}

More recently, an alternate approach has become popular that solves these
issues. \textcite{montague_hyperscanning_2002} named it `hyperscanning'. As of
May 2022, there are over 3530 publications on hyperscanning\footnote{As
determined by a Google Scholar search for the term `hyperscanning'.}. Most of
those were written very recently: 2040 of them were published in 2018 or later.
Hyperscanning is defined as recording brain imaging data for two (or more)
participants simultaneously. These two participants are also called a dyad. This
allows us to treat the dyad's brains as a single entity `coupled' through their
respective perceptual and motor systems \parencite{hasson_brain--brain_2012}.

Social interaction is studied with hyperscanning in a large variety of ways
\parencite{czeszumski_hyperscanning_2020}. Often, studies are performed in the
lab where conditions can be precisely controlled. Some of those only allow
interaction through a computer interface, as in the prisoner's dilemma studies
of \textcite{hu_inter-brain_2018,de_vico_fallani_defecting_2010}. Other studies
strictly control the task but allow participants to see each other either
through video links \parencite{dumas_inter-brain_2010,schippers_mapping_2010}
or directly while interacting using gestures
\parencite{yun_interpersonal_2012}. On the other hand, some studies record
participants in a more naturalistic setting for the activities they perform,
like in the classroom \parencite{dikker_brain--brain_2017} or the monastery
\parencite{van_vugt_inter-brain_2020}. While having complete control allows for
more precise conclusions, \textcite{konvalinka_two-brain_2012} argue that
emergent patterns could arise in the brain as a result of social interaction,
leading to a difference in experiments where participants are to some extent
observers \parencite[is a good example; it is difficult to avoid in fMRI
studies]{schippers_mapping_2010} compared to where they actively interact. The
same can be said about being able to interact in person or through a computer
\parencite{konvalinka_two-brain_2012}. \textcite{liu_clarifying_2014} categorize
hyperscanning tasks along three dimensions: whether they require concurrent
body movement or the participants interact only in a turn-based fashion, whether
participants compete or cooperate and whether the participants can influence
each other while the task is ongoing or not. If they can influence each other
the task is called interdependent, otherwise it is independent.

\subsection{Inter-brain synchrony}

When analysing brain data acquired using hyperscanning, the most common strategy
is to look for inter-brain synchrony
\parencite[IBS;][]{ayrolles_hypyp_2021}. IBS occurs when there are functional
similarities in the brain activity of individuals. IBS is often found in the
brain data of such individuals when they socially interact
\parencite{konvalinka_two-brain_2012}. While the hyperscanning approach is most
often used to collect brain data, synchrony has also been found in other
physiological signals including ``heart rate\footnote{See
also \textcite{mccraty_new_2017}.}, pupil size, gaze position and saccade rate''
\parencite{madsen_cognitive_2022}. \textcite{novembre_hyperscanning_2021} argue
hyperscanning alone cannot tell us whether IBS is required for social
interaction or if it just co-occurs with it, but extending the paradigm to
include multi-brain stimulation could make that possible.

What exactly causes IBS has not yet been firmly established
\parencite{liu_interactive_2018}, but suggested causes include common cognitive
processing \parencite{hamilton_hyperscanning_2021,madsen_cognitive_2022},
shared observations \parencite{hamilton_hyperscanning_2021}, and more generally
shared attention \parencite{dikker_brain--brain_2017,sebanz_joint_2006}. How
these processes in turn result in synchronized oscillations is also still mostly
unknown \parencite{liu_interactive_2018}, but this too is an area of active
research \parencite{hamilton_hyperscanning_2021,koike_hyperscanning_2015}.

We measure IBS using different functional connectivity measures, all of which
calculate the similarity between the brain signals recorded for both (or more)
participants in a specific way \parencite{czeszumski_hyperscanning_2020}. These
measures were originally developed to study connectivity within a single system
or brain \parencite{babiloni_social_2014}. A nice overview of them from that
perspective is given by \textcite[section~5]{cohen_analyzing_2014}. As
inter-brain data has different properties than intra-brain data, their
interpretation when used to calculate IBS instead of intra-brain synchrony is
different and complex \parencite{ayrolles_hypyp_2021}. For example, when
interpreting intra-brain synchrony, you need to be careful to not interpret a
single signal measured at multiple points due to volume conduction as synchrony
\parencite{czeszumski_hyperscanning_2020}. That is not an issue when the signals
are coming from different participants. On the other hand, while intra-brain
synchrony is driven by the anatomical structure of the brain
\parencite{ayrolles_hypyp_2021,dumas_anatomical_2012}, IBS can only occur
through ``an indirect chain of events'' \parencite{babiloni_social_2014}, as (of
course) no direct communication can occur between brains as opposed to brain
regions \parencite{babiloni_social_2014}. IBS is driven by sensorimotor coupling
instead, which is a less reliable mechanism \parencite{dumas_anatomical_2012}.

Different measures focus on different kinds of similarities
(i.e. different aspects of oscillations) in the brain signals. We will see
examples of this in the simulation section. Because of that,
\textcite{czeszumski_hyperscanning_2020} argues that it is misleading to refer
to them all with the umbrella term `inter-brain synchrony'.

It is important to keep in mind that many factors can influence IBS before
interpreting its results. \textcite{cheng_synchronous_2015} found
an effect of gender: more synchrony was found in male-male dyads than
female-male dyads, which in turn had higher IBS than female-female ones.
The relationship between the participants is also important.
\textcite{dikker_crowdsourcing_2021} found a positive correlation between
relationship duration and IBS, and \textcite{pan_cooperation_2017} found more
IBS between lovers than friends or strangers. Less IBS has been found in
individuals with    autism spectrum disorder
\parencite[ASD;][]{salmi_brains_2013,valencia_what_2020}.

Due to the properties of the signals of different brain imaging methods,
they each have their own classes of IBS measures that are often used alongside
them \parencite{babiloni_social_2014}. For example, when working with
EEG hyperscanning data frequency domain-based measures are often used, while
temporal correlations are more suited to functional magnetic resonance imaging
(fMRI) data \parencite{babiloni_social_2014}. This is due to the lower temporal
resolution of the latter \parencite{czeszumski_hyperscanning_2020}.

IBS is often found in interacting
partners in ``prefrontal and centro-parietal brain areas [...] across a wide
range of frequencies, including delta, theta, alpha, beta and gamma''
\parencite{konvalinka_two-brain_2012}. In \textit{prisoner's dilemma} studies,
less synchrony is found in the alpha and theta bands when participants defect
than when they cooperate \parencite{valencia_what_2020,hu_inter-brain_2018,de_vico_fallani_defecting_2010}.
\citeauthor{de_vico_fallani_defecting_2010} additionally found the same effect
in the beta and gamma bands, and were able to succesfully predict whether a user
will defect in an iterated prisoner dilemma task based on IBS data.

\subsubsection{Analysis methodologies}

Little research has gone into which methodology to adopt when researching
IBS. Simple connectivity measures like the phase locking value
\parencite[PLV;][]{lachaux_measuring_1999} have been most popular
\parencite{czeszumski_hyperscanning_2020,burgess_interpretation_2013}.
The first systematical comparison of the performance of a number of
measures in a hyperscanning context was done by
\textcite{burgess_interpretation_2013}. \citeauthor{burgess_interpretation_2013}
found that a number of measures, including PLV, suffered from detecting
spurious connections in simulations. Instead,
\citeauthor{burgess_interpretation_2013} recommends using the more robust
circular correlation (CCorr) and Kraskov mutual information measures.

\textcite{burgess_interpretation_2013} concludes that ``different people
presented with the same conditions will produce similar EEG responses'',
regardless of whether they were interacting. This can be somewhat mitigated by
using measures like the imaginary part of coherency
\parencite[ImagCoh;][]{nolte_identifying_2004,yoshinaga_comparison_2020},
which will ignore signals that are in phase (or anti-phase) with each other
\parencite[p.~346]{cohen_analyzing_2014}. An example of such a signal would be
brain activity in the sensory cortex caused by a (strong) stimulus
\parencite{dikker_crowdsourcing_2021}. The ImagCoh measure
was originally developed to counteract volume conduction, but as this is not an
issue with hyperscanning it is useless in that respect
\parencite{ayrolles_hypyp_2021}.

\textcite{ayrolles_hypyp_2021} recently made a push for standardization in IBS
calculation by making a complete hyperscanning analysis pipeline available.
\citeauthor{ayrolles_hypyp_2021} also advise to use amplitude based measures
like power correlation when interested in neural states and phase-based measures
when studying more fine-grained cognitive processes.

\subsection{Research questions}

The hard work of \textcite{burgess_interpretation_2013,ayrolles_hypyp_2021}
notwithstanding, it is clear that only a little is known about the consequences
of varying parts of an IBS analysis. The same is true for the
interpretation of IBS measures in a hyperscanning context. Because of this gap
in the literature, the aim of this graduation project is to investigate the
sensitivity of IBS calculations in a social coordination task to different
connectivity measures and other methodological choices.

To clarify the interpretation of different IBS measures, we calculate and
compare their values on (simple) artificial data. This allows us to see what
kind of patterns in the data they respond to. Additionally, we develop a method
that generates synthetic data for a given IBS value. This method can be used
to perform power analyses for IBS experiments. We focus on the PLV, ImagCoh and
CCorr measures.

To investigate the sensitivity to methodological choices under realistic
conditions, we perform a number of IBS analyses on a cooperative, turn-based and
interdependent task. First, we analyse the effect of different time windows of
interest and frequency analysis calculation methods on these values. Second, we
analyse whether significant IBS is present in the emperical data using different
permutation tests. Third, we inspect IBS values over time. Finally, we
attempt to predict peformance in the task based on the IBS values. We
vary the prediction scenarios and classification methods.

As most of the variations we make should still lead to the same result, we
hypothesize that the analysis is robust to such changes in methodology.
Based on \textcite{burgess_interpretation_2013}'s previous work, we
expect PLV to perhaps return more spurious results than the CCorr measure. Our
expectations regarding the ImagCoh measure are more nuanced. It is both a phase-
and amplitude based measure, allowing it to potentially pick up on effects that
the other phase-based measures might miss. But it might also miss in-phase IBS
detected by the other measures.
